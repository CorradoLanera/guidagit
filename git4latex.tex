% % % % % % % % % % % % % % % % % % % % % % % % % % % % % % % % % % % % % % % % 
% 
% LICENZA
% 
% Quest'opera è stata rilasciata sotto la licenza
% Creative Commons Attribution-NonCommercial-ShareAlike 2.5 Italy.
% Per leggere una copia della licenza visita il sito web
% http://creativecommons.org/licenses/by-nc-sa/2.5/it/
% o spedisci una lettera a Creative Commons, 171 Second Street, Suite 300, San
% Francisco, California, 94105, USA.
%
% This work is licensed under the Creative Commons
% Attribution-NonCommercial-ShareAlike 3.0 Unported License. To view a copy of
% this license, visit http://creativecommons.org/licenses/by-nc-sa/3.0/ or send
% a letter to Creative Commons, 171 Second Street, Suite 300, San Francisco,
% California, 94105, USA.
%
% For info send a mail to pietro.giuffri@gmail.com
% 
% % % % % % % % % % % % % % % % % % % % % % % % % % % % % % % % % % % % % % % %

\documentclass[a4paper,12pt,oneside]{article}
\usepackage[utf8]{inputenc}
\usepackage[italian]{babel}

\usepackage[dvipsnames, usenames]{xcolor}
\definecolor{GY}{named}{GreenYellow} % SkyBlue

\usepackage{listings}
\usepackage{fourier}

\usepackage[bookmarks=false]{hyperref}
\usepackage{breakurl}
\usepackage{guit}

% \usepackage[%
% paper size={90mm,120mm},left=2mm,right=2mm,top=2mm,bottom=2mm,nohead]%
% paper size={200mm,200mm},left=20mm,right=20mm,top=10mm,bottom=15mm,nohead]%
% {geometry}

\begin{document}
\lstset{%
%   frame=single,
  backgroundcolor=\color{GY},
  breaklines = true, % per mandare a capo le righe troppo lunghe
  breakautoindent = true, % indenta le righe spezzate
  breakindent = 200pt, % indenta le righe di 30pt
  language=bash,
  showspaces=false,
  showstringspaces=false,
  showtabs=false,
  emph={touch,mkdir,echo,rm,git},
  emphstyle=\textbf
  }

\title{git4\LaTeX \\
  Una guida introduttiva a git per progetti \LaTeX}
\author{Pietro Giuffrida}
\maketitle
\newpage
\tableofcontents
\newpage
% \large
\section{Scopo della guida}
Git è un revision control system, vale a dire un programma che permette di
tener traccia di tutte le modifiche e le evoluzione effettuate nel corso della
stesura di un codice o di un qualsiasi progetto su supporto digitale. git è
rilasciato sotto licenza GPL, ed è disponibile, oltre che nei repository delle
varie distribuzioni GNU-linux, all'indirizzo \url{http://git-scm.com/}.

\LaTeX è un programma di composizione tipografica di alta qualità. \LaTeX è
free-software. Generalmente fa parte dei programmi fondamentali preinstallati in
ogni distribuzione GNU-linux, ed è comunque disponibile a partire dall'indirizzo
\url{http://www.latex-project.org/} per i principali OS.

In questo documento desidero mostrare come utilizzare git per tener traccia
delle modifiche rilevanti e delle versioni elaborate nel corso dell'elaborazione
di un documento scritto con \LaTeX. Tramite git è infatti possibile mantenere un
backup incrementale del proprio codice sorgente, sia in locale che in remoto,
senza incorrere in un eccessivo dispendio di energie e senza deconcentrarsi
eccessivamente dalla stesure del proprio testo. I passaggi descritti
relativamente all'uso di git dovrebbero essere validi in linea di principio per
l'elaborazione di qualsiasi documento (foto, codice sorgente, file doc, xls
e quant'altro), anche se la guida è strutturata in modo specifico per
illustrare l'interazione con \LaTeX.

Solo il primo paragrafo è indispensabile. In esso vengono mostrati i passaggi
fondamentali per la creazione di un repository locale del proprio progetto, per
il salvataggio progressivo delle versioni, e quindi per svolgere eventuali
operazioni di ripristino.

I paragrafi successivi sono invece dedicate alla creazione ed alla
sincronizzazione di un repository remoto (che si tratti di un servizio on-line
o semplicemente di una memoria esterna), e ad uno script bash che permette di
eseguire dei commit automatici ad ogni salvataggio dei file.

La guida non si prefigge alcun compito specifico per quanto riguarda \LaTeX, a
proposito del quale esistono numerose guide\footnote{Si veda a questo proposito,
oltre alla documentazione reperibile sul proprio computer mediante il comando
``texdoc nomepacchetto'', la documentazione in italiano reperibile a partire dal
sito del \guit\ (\url{http://www.guit.sssup.it}), oltre all'ottima guida di L. Pantieri
disponibile sul sito \url{http://www.lorenzopantieri.net/LaTeX.html}}.

La guida non contempla l'uso di gui o di qualunque interfaccia grafica. Tutti i
comandi ed i passaggi illustrati sono eseguiti tramite una shell linux.
Non ho esperienza né di git né di \LaTeX su sistemi operativi non
unix-like. Un minimo di compatibilità con altri sistemi è garantita dal fatto
che i comandi tipici della shell linux sono evidenziati e commentati, in modo
che l'utente di altri OS o abituato all'uso di GUI possa sostituirli svolgendo
altrimenti le medesime attività. I comandi di git dovrebbero invece restare i
medesimi in ogni OS, per quanto anche in questo caso le medesime attività
possano essere svolte mediante GUI.

Non sono un esperto di informatica, ma trovo bello far le cose, per quanto
possibile, con le mie mani, sapere cosa fa la macchina ed avere l'illusione che
nel rapporto quasi-simbiotico con il computer sia io a decidere.

Dato il carattere ludico della presente guida, scelgo come licenza la Creative
Commons Attribution-NonCommercial-ShareAlike 2.5 Italy. Gli unici vincoli sono
quindi quelli di riconoscere la paternità dell'opera
originale, di non lucrare sulla sua distribuzione, e di distribuire l'opera ed
i suoi eventuali derivati con la medesima licenza\footnote{
Quest'opera è stata rilasciata sotto la licenza
Creative Commons Attribution-NonCommercial-ShareAlike 2.5 Italy.
Per leggere una copia della licenza visita il sito web
\url{http://creativecommons.org/licenses/by-nc-sa/2.5/it/}
o spedisci una lettera a Creative Commons, 171 Second Street, Suite 300, San
Francisco, California, 94105, USA.}.

\section{Un semplice progetto \LaTeX}
Vogliamo sviluppare un documento con \LaTeX, utilizzando git come CVS.
Git non funziona in modo particolarmente esotico. Si tratta semplicemente di
creare una directory, di posizionare in essa i nostri file .tex e di dire a git
di considerare tale directory come un repository.

\subsection{Creazione ed inizializzazione del repository}
\begin{lstlisting}
~$ mkdir progetto
~/progetto$ cd progetto
~/progetto$ touch np_main.tex
~/progetto$ git init
~/progetto$ git add .
~/progetto$ git commit -am "Inizializzazione del nuovo progetto"
\end{lstlisting}

Vediamo con calma i singoli passaggi. I primi tre comandi servono
rispettivamente per creare la directory (mkdir nome\_directory), per spostarsi
al suo interno (cd nome\_directory) e per creare un file vuoto chiamato
np\_main.tex (touch nome\_file).

I passaggi successivi, eseguiti sempre dati dall'interno della directory, sono
quelli specifici di git. Con essi effettuiamo:
l'inizializzazione della directory come un repository git (git init); l'aggiunta
al repository appena creato di tutti i file al momento presenti nella stessa
directory (git add .); il salvataggio di una prima versione del progetto (commit
-am "nota di versione").

Così facendo saremmo già pronti a lavorare con il nostro editor di fiducia sul
file .tex appena creato.
Non resta che eseguire un commit al termine di ogni modifica rilevante, in
modo da salvare una determinata versione del progetto.

\begin{lstlisting}
~/progetto$ echo "una modifica rilevante" >> np_main.tex
~/progetto$ git commit -am "ulteriore sviluppo"
\end{lstlisting}

\bigskip
\subsection{Aggiungere altri file al progetto: il file `.gitignore'}
Finora abbiamo eseguito il commit passando a git le due opzioni ``-am''. Con
l'opzione ``-m'' diciamo a git di utilizzare il testo tra virgolette che segue
il comando come nota di versione del commit che stiamo eseguendo. Con l'opzione
``-a'' chiediamo a git di salvare tutte le modifiche apportate a tutti i file
che abbiamo precedentemente aggiunto al progetto.

Si presenta ora uno scenario tra i più comuni: dobbiamo aggiungere un ulteriore
file al progetto, per esempio un file .tex contenente un ulteriore capitolo, o
un file .bib contenente al bibliografia, o un qualsiasi altro file.
Quando eseguiamo il commit successivo, potremmo aspettarci che git si accorga
del nuovo arrivato, ed in effetti dovrebbe restituire un messaggio di questo
tipo:

\begin{lstlisting}
~/progetto$ touch np_secondary.tex
~/progetto$ git commit -am "aggiunto file np_secondary.tex"
# On branch master
# Untracked files:
#   (use "git add <file>..." to include in what will be committed)
#
#       np_secondary.tex
nothing added to commit but untracked files present (use "git add" to track)
\end{lstlisting}

Git si accorge quindi della presenza di un nuovo file, ma non lo aggiunge
automaticamente al progetto. D'altra parte, se avesse rilevato delle modifiche
ai file precedentemente aggiunti al progetto, non si sarebbe nemmeno curato di
comunicarci che il nuovo file non è ancora stato aggiunto al progetto. Si
sarebbe infatti limitato a salvare le modifiche ai file che gli abbiamo
precedentemente detto di gestire. Solo con il comando ``git status'', git ci
comunica con certezza l'anomalia di un nuovo file non ancora caricato nel
repository.

In tale circostanza si può scegliere di procedere in due modi:
aggiungere indiscriminatamente tutti i file della directory al progetto che
stiamo sviluppando; oppure aggiungere un singolo file. Nel primo caso dovremmo
ripetere il comando già usato in fase di inizializzazione:

\begin{lstlisting}
~/progetto$ git add.
\end{lstlisting}

Nel secondo caso aggiungiamo un singolo file:

\begin{lstlisting}
~/progetto$ git add np_secondary.tex
\end{lstlisting}

Per quanto possa sembrare eccessivo, io preferisco usare il secondo comando.
Mi accade con estrema frequenza di
dover aggiungere dei file ad un progetto, e spesso sono fin troppo
distratto da quel che sto scrivendo per occuparmi di quel git si aspetta da me
L'aggiunta indiscriminata di ogni file nella directory al repository presenta
però delle controindicazioni.
La più ovvia per chi lavora con \LaTeX è la seguente:
nel corso dell'elaborazione di un testo inevitabilmente si
procederà alla generazione del documento in pdf a partire dai sorgenti;
\LaTeX provvederà quindi alla generazioni di una serie di file secondary (.toc,
.out, ...), nonché di un pdf più o meno inutile.
Se eseguissimo il comando ``git add .'' subito dopo una compilazione,
evidentemente git aggiungerebbe al repository anche i vari file di lavoro, dal
pdf, ai file  di log. Per ovviare a questo inconveniente, e continuare
pigramente ad eseguire ``git add .'', è utile creare nella nostra directory di
un file denominato ``.gitignore''.

\begin{lstlisting}
~/progetto$ touch .gitignore
~/progetto$ echo "*.log" >> .gitignore
~/progetto$ echo "*.pdf" >> .gitignore
~/progetto$ echo "*.blg" >> .gitignore
~/progetto$ echo "*.bbl" >> .gitignore
~/progetto$ echo "*.aux" >> .gitignore
~/progetto$ echo "*-blx.bib" >> .gitignore
~/progetto$ echo "*.out" >> .gitignore
~/progetto$ echo "*~" >> .gitignore
~/progetto$ git add .gitignore
~/progetto$ git commit -am "Aggiunto il file .gitignore"
\end{lstlisting}

Mediante questo file, istruiamo il repository a proposito di tutti i
file che non fanno effettivamente parte del progetto, pur essendo presenti
nella directory.
Da questo momento in poi git si limiterà ad ignorarli - il che ci permette di
eseguire prudenzialmente il comando ``git add.'' prima di ogni commit.

\subsection{Consultazione dei log e ripristino di file}
Ipotizziamo a titolo esemplificativo il seguente scenario:
abbiamo accidentalmente eliminato un file del nostro
progetto, ed abbiamo anche eseguito un commit.

\begin{lstlisting}
~/progetto$ touch np_secondary.tex
~/progetto$ echo "pippo" >> np_secondary.tex
~/progetto$ git add .
~/progetto$ git commit -am "Ho solo iniziato a lavorare"
~/progetto$ rm np_secondary.tex
~/progetto$ git commit -am "Ma ho gia' perso tutto"
\end{lstlisting}

Ora chiediamo conto a git della sua capacità di ripristinare una versione
precedentemente salvata. La prima cosa da fare è consultare i log dei commit
precedentemente effettuati, per decidere quale di essi ripristinare. Il comando
appropriato sarebbe quindi ``git log'', che contempla anche un'ampia serie di
interessanti opzioni. Rimandando al manuale per la maggior parte di tali
opzioni, segnalo solo che è possibile:
\begin{itemize}
\item selezionare esclusivamente i log relativi ad un dato file, con la
sintassi ``git log nome\_file''
\item chiedere a git di stampare solo gli ultimo \textit{n} log, con la
sintassi ``git log -n \textit{n}''
\end{itemize}

Trattandosi di un file eliminato, non possiamo chiedere a git di stampare solo i
log ad esso relativi. Dobbiamo quindi cercare tra i log più recenti fino ad
identificare uno stadio del progetto in cui il file è ancora presente.

\begin{lstlisting}
~/progetto$ git log -n 2
commit 380228784f4095efdbce9d1d3408bcbdf01548cb
Author: Pietro Giuffrida <pietro.giuffri@gmail.com>
Date:   Sat Oct 2 18:58:44 2010 +0200

    Ma ho gia' perso tutto

commit c21d825df42f55ba127487fd93acf1c13c41b028
Author: Pietro Giuffrida <pietro.giuffri@gmail.com>
Date:   Sat Oct 2 18:55:07 2010 +0200

    Ho solo iniziato a lavorare
\end{lstlisting}

Nel nostro caso, l'eliminazione del file è ben documentata dalle note, e
possiamo quindi subito verificare il contenuto del commit facendo riferimento
alla stringa alfanumerica che lo identifica univocamente.

\begin{lstlisting}
~/progetto$ git show c21d825df42f55ba127487fd93acf1c13c41b028
commit c21d825df42f55ba127487fd93acf1c13c41b028
Author: Pietro Giuffrida <pietro.giuffri@gmail.com>
Date:   Sat Oct 2 18:58:29 2010 +0200

    aggiunto file np_secondary.tex

diff --git a/np_secondary.tex b/np_secondary.tex
new file mode 100644
index 0000000..bfa5424
--- /dev/null
+++ b/np_secondary.tex
@@ -0,0 +1 @@
+pippo
\end{lstlisting}

Il file np\_secondary.tex fa effettivamente parte del commit da noi
individuato. Esso è identificato da un'ulteriore stringa alfanumerica
denominata index. Git, tramite il comando diff, ci comunica inoltre che esso
contiene la stringa ``pippo''.
Possiamo quindi procedere a ricrearne una copia facendo riferimento all'index

\begin{lstlisting}
~/progetto$ git show bfa5424 > file_ripristinato
\end{lstlisting}

La procedura, con le complicazioni del caso, dovrebbe essere valida anche per
circostanze concrete. Vi consiglio comunque di fare delle prove simulando per
esempio commit di più file contenenti varie modifiche, per far pratica con gli
indici e la sintassi di git.
Possiamo ipotizzare in questo caso uno scenario di questo tipo:

\begin{lstlisting}
~/progetto$ touch topolino pippo
~/progetto$ echo qualcosa >> topolino
~/progetto$ echo qualcosaltro >> pippo
~/progetto$ git add.
~/progetto$ git commit -am "creati e modificati topolino e pippo"
~/progetto$ echo "ancora qualcosa" >> topolino
~/progetto$ git commit -am "ulteriormente elaborato topolino"
~/progetto$ git log topolino
commit 87548d4cd43cda917cf788866abd76f8399b639b
Author: Pietro Giuffrida <pietro.giuffri@gmail.com>
Date:   Sat Oct 2 19:27:23 2010 +0200

    ulteriormente elaborato topolino

commit f7a696e891bb81446f56adc996bc0a20a23ae9b5
Author: Pietro Giuffrida <pietro.giuffri@gmail.com>
Date:   Sat Oct 2 19:27:01 2010 +0200

    creati e modificati topolino e pippo

~/progetto$ git show f7a696e891bb81446f56adc996bc0a20a23ae9b5
commit f7a696e891bb81446f56adc996bc0a20a23ae9b5
Author: Pietro Giuffrida <pietro.giuffri@gmail.com>
Date:   Sat Oct 2 19:27:01 2010 +0200

    creati e modificati topolino e pippo

diff --git a/pippo b/pippo
new file mode 100644
index 0000000..a09d8a1
--- /dev/null
+++ b/pippo
@@ -0,0 +1 @@
+qualcosa
diff --git a/topolino b/topolino
new file mode 100644
index 0000000..a09d8a1
--- /dev/null
+++ b/topolino
@@ -0,0 +1 @@
+qualcosaltro

~/progetto$ git show a09d8a1 > topolino_ripristinato
\end{lstlisting}

Chiaramente, git permette il ripristino esclusivamente dei file aggiunti al
repository, ed esclusivamente di versioni esplicitamente salvate mediante il
comando commit. I salvataggi effettuati tra un commit e l'altro non sono quindi
ricostruibili.

\subsection{Configurazioni basilari di git}
Per quanto non strettamente indispensabile per un uso individuale di git sul
proprio pc, segnalo alcune configurazioni elementari necessarie per un uso
ottimale di git.

In primo luogo occorre passare a git qualche informazione circa l'utente.
\begin{lstlisting}
git config --global user.name "Pietro Giuffrida"
git config --global user.email pietro.giuffri@gmail.com
\end{lstlisting}

In secondo luogo è utile dire a git di colorare i log, in modo da renderli più
leggibili.
\begin{lstlisting}
git config --global color.branch auto
git config --global color.diff auto
git config --global color.interactive auto
git config --global color.status auto
\end{lstlisting}

\section{Backup}
\subsection{Backup su periferica esterna (usb)}
\begin{lstlisting}
~/progetto$ git clone -l file:///home/pietro/elab/git4latex/ /media/usb/gigt/
\end{lstlisting}

\subsection{Backup on-line}

\begin{enumerate}
\item registrarsi su gitorius (richiede ssh key e passphrase!)
\item git remote add origin git@gitorious.org:git4latex/git4latex.git
(che chiederà la passphrase!)
\item git push origin master
per sincronizzare il tutto di volta in volta a fine giornata
\end{enumerate}

Prima sincronizzazione da locale verso gitorius:
\begin{lstlisting}
$ git remote add origin git@gitorious.org:progetto/progetto.git
\end{lstlisting}

A fine giornata di lavoro:
\begin{lstlisting}
$ git push origin master
\end{lstlisting}

Se volete ricreare il repository in locale:
\begin{lstlisting}
$ mkdir riprogetto
$ cd riprogetto
$ git clone git://gitorious.org/git4latex/git4latex.git
\end{lstlisting}

\section{commit automatico: inotifywait}

\begin{lstlisting}
inotifywait -q -m -e CLOSE_WRITE --format="git commit -m 'autocommit on change' %w" file.txt | sh
\end{lstlisting}

\begin{lstlisting}
#!/bin/bash
#
# gitwait - watch file and git commit all changes as they happen
#

while true; do

  inotifywait -qq -e CLOSE_WRITE ~/.calendar/calendar

  cd ~/.calendar; git commit -a -m 'autocommit on change'

done
\end{lstlisting}

ma è quasi pericoloso se salvi automaicamente ogni tot minuti!!



\end{document}
