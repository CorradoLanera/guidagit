\documentclass{article}
\usepackage[utf8]{inputenc}
\usepackage[italian]{babel}

\usepackage{listings}
\usepackage{fourier}

\usepackage{color}
\usepackage{listings}
\usepackage{framed}


\begin{document}
\title{git4\LaTeX \\
  Una guida introduttiva a git per progetti \LaTeX}
\author{Pietro Giuffrida}
\maketitle

La presente guida si propone di presentare sinteticamente ed in modo spero non
troppo lacunoso l'uso che si può fare di git nel corso di un lavoro sviluppato
in \LaTeX\. Ritengo che git possa rappresentare un programma di estrema utilità
per mantenere in ordine le versioni progressivamente accumulate di uno o più
documenti. git permette infatti di gestire le versioni, il loro recupero o la
loro consultazione, il backup locale e remoto, tutto in estrema economia.

Usando esclusivamente linux, il terminale, kate come editor!


\section{Un semplice progetto \LaTeX}
Vogliamo sviluppare un documento con \LaTeX, utilizzando git come CVS.
Aperto un terminale, i passaggi indispensabili sono:
\begin{enumerate}
\item creiamo la cartella che conterrà i nostri sorgenti;
\item creiamo il primo file .tex;
\item inizializziamo la medesima directory come un repository git;
\item diciamo a git quali sono i file da \textit{tenere d'occhio},
aggiungendoli al repository.
\end{enumerate}

\begin{framed}
\lstinputlisting{git_command.sh}
\end{framed}

Così facendo saremmo già pronti a lavorare con il nostro editor di fiducia, e
non resterebbe che eseguire un commit al termine di ogni modifica rilevante, in
modo da salvare cioè una determinata versione del progetto.

\begin{framed}
\begin{lstlisting}
commit -am "un ulteriore sviluppo del progetto"
\end{lstlisting}
\end{framed}

\subsection{Aggiungere altri file al progetto}
Finora abbiamo eseguito il commit passando a git le due opzioni "-am".
Con l'opzione "-m" diciamo a git di utilizzare il testo virgolettato che segue
il comando come nota di versione del commit che stiamo eseguendo.
Con l'opzione "-a" chiediamo a git di salvare tutte le modifiche apportate a
tutti i file che abbiamo precedentemente aggiunto al progetto.

Si presenta ora uno scenario tra i più comuni: dobbiamo aggiungere un ulteriore
file al progetto, per esempio un file .tex contenente un ulteriore capitolo, o
un file .bib contenente al bibliografia, o un qualsiasi altro file.
Quando eseguiamo il commit, potremmo aspettarci che git si accorga del nuovo
arrivato, ed in effetti dovrebbe restituire un messaggio di questo tipo:

\begin{framed}
\begin{lstlisting}
git commit -am "aggiunto file secondary.tex"
# On branch master
# Untracked files:
#   (use "git add <file>..." to include in what will be committed)
#
#       secondary.tex
nothing added to commit but untracked files present (use "git add" to track)
\end{lstlisting}
\end{framed}

git si accorge quindi della presenza di un nuovo file, ma non lo aggiunge
automaticamente al progetto. D'altra parte, se avesse rilevato delle modifiche
ai file precedentemente aggiunti al progetto, non si sarebbe nemmeno curato di
comunicarci che il nuovo file non è ancora stato aggiunto al progetto.
Si potrebbe quindi scegliere di procedere in due modi: aggiungere
indiscriminatamente tutti i file della directory al progetto che stiamo
sviluppando; oppure aggiungere un singolo file.
Nel primo caso dovremmo ripetere il comando già usato in fase di
inizializzazione:

\begin{framed}
\begin{lstlisting}
git add.
\end{lstlisting}
\end{framed}

Nel secondo caso aggiungiamo un singolo file:

\begin{framed}
\begin{lstlisting}
git add secondary.tex
\end{lstlisting}
\end{framed}

Per quanto possa sembrare eccessivo, dato che accade con estrema frequenza di
dover aggiungere dei file ad un progetto, e che spesso si è fin troppo
distratti da quel che si sta scrivendo per occuparci di quel che fa git, credo
che sia preferibile aggiungere eseguire la prima delle due operazioni.

Dato però che nel corso dell'elaborazione di un testo inevitabilmente si
procederà alla generazione del documento in pdf mediante i vari comandi di
\LaTeX, si potrebbe determinare un controsenso. Se infatti eseguiamo il comando
"git add ." dopo aver eseguito,
per esempio, pdflatex, evidentemente git aggiungerebbe al repository anche i
vari file di lavoro, dal pdf, ai file .log, .out e vari altri.
Per ovviare a questo inconveniente, è utile creare nella nostra directory di
lavoro un file denominato ".gitignore".

\begin{framed}
\begin{lstlisting}
touch .gitignore
echo "*.log" >> .gitignore
echo "*.pdf" >> .gitignore
echo "*.blg" >> .gitignore
echo "*.bbl" >> .gitignore
echo "*.aux" >> .gitignore
echo "*-blx.bib" >> .gitignore
echo "*.out" >> .gitignore
echo "*~" >> .gitignore
git add .gitignore
git commit -am "Aggiunto il file .gitignore"
\end{lstlisting}
\end{framed}

Mediante questo file, diciamo a git che tutti i file con le estensioni in esso
definite non devono essere considerati appartenenti al progetto, e che non si
deve dar pena se non li aggiungiamo al repository.

\section{Consultazione dei log e ripristino di file}
Presi da una strano forma di pazzia abbiamo eliminato un file del nostro
progetto.



\end{document}